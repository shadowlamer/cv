\documentclass [a4paper,10pt]{article}
\usepackage[T1]{fontenc}
\usepackage[utf8]{inputenc}
\usepackage[russian]{babel}
\usepackage[hidelinks]{hyperref}
\usepackage{marvosym}


\renewcommand{\labelitemi}{}

\begin{document}
\begin{center}
{\scshape\LARGE Вадим \par}
\Letter\hspace{6pt}\href{mailto:sl@anhot.ru}{sl@anhot.ru},
\Mobilefone\hspace{6pt}\href{tel:+79638704302}{+7 963 870 43 02}

разработчик
\end{center}
\section*{Значимый опыт}
\paragraph{Успешные\protect\footnote{выжили и продолжают существовать} стартапы}
\begin{description}
\item[<<Третий кран>>] \url{https://3voda.ru/} (c 2017) 

Полностью несу ответственность за всю электронику и ПО.

\item[<<INSYTE Electronics>>] \url{https://insyte.ru/} (2003 -- 2008)

Начинался с моего курсача про передачу данных по силовой линии. В 2007 -- 2008 принимал непосредственное участие в работе.
Настоял на переводе на <<честный>> ModBus, Добился идемпотентной реализации функций. Разработал часть устройств из линейки <<LanDrive2>>.
Занимался запуском контрактного производства, готовил документацию, тестовые стенды. Осуществлял техподдержку.
\end{description}

\paragraph{Не такие успешные стартапы}
\begin{description}
\item[<<Magicscope>>] \url{https://magicscope.com} (c 2009)

Вот тут можно почитать --- \url{https://perm.aif.ru/society/people/youtube_otdyhaet_permyaki_sozdali_novuyu_mediaplatformu}
	
Мои --- архитектура, все что касается медиасервера и конвертирования видео, часть бэка, весь фронт и мобильное приложение.
\end{description}

\paragraph{Работа в команде}
\begin{description}
\item[<<PARMA Technologies Group>>] \url{https://parma.ru/} (2016 --2017)
	
Веб, энтерпрайз, фулл стэк. Делал что скажут. Поучаствовал в десятке различных проектов. 
Из тех, что официально упомянуты на сайте --- <<На дальний восток>>, остальные примерно такого же масштаба.

\item[<<Региональный центр автоматизации>>] (2008 -- 2009)

Пытались с коллегами построить свою галеру. Занимались видеонаблюдением, вибродиагностикой, промышленной автоматизацией, немного вебом.
\end{description}

\paragraph{Преподавательская деятельность}
\begin{description}
\item[ПГИИК] \url{http://psiac.ru/} (2009 -- 2013)

Старший преподаватель, кафедры ИИТ. Читал лекции и вел лабораторные работы по мультимедийным технологиям и физике.

\item[ПГТУ] \url{https://pstu.ru/} (2004 - 2007)

Ассистент, кафдры ИТАС. Вел лабораторные работы по различным околокомпьютерным дисциплинам, 
участвовал в НИР кафедры, работе Российско-австрийского центра технологий fieldbus\footnote{тогда это не было мейнстримом, а сейчас называется <<интернет вещей>>}.
\end{description}

\section*{Применяю}
\begin{description}
\item[Электроника:]
	Сначала рисовал, как все, в PCAD, в середине 2000х перешел на Eagle, соблазнившись кросплатформенностью 
	и библиотекой компонентов. Сейчас в основном использую EasyEDA. Есть опыт использования KiCAD.
	Осциллограф, паяльник, напильник, само собой. Обычно делаю решения на STM32, ранее на AVR, но могу и чисто аналоговые.

\item[Встраиваемое ПО:]
	С, С++, BASH. В качестве IDE в разное время использовал Code::Blocks, Eclipse, Keil uVision. 
	Доводилось использовать IAR. Сейчас подсел на CLion.

\item[Системное ПО:]
	Linux (преимущественнно Debian) и все что связано (C, C++, BASH). Могу собрать из спичек и желудей дистрибутив под железку. 
	Есть относительно успешный опыт написания драйверов, загрузчика. 

\item[Прикладное ПО:]
	Пока не определился с подходящей технологией для быстрого написания прикладного ПО. 
	Использовал Delphi/Kylix/Lazarus, Electron. Сейчас пробую Qt.

\item[Веб/бэкенд:]
	Java, Spring, Spring boot. IDE --- ранее --- Eclipse, сейчас --- IntelliJ IDEA.

\item[Веб/фронтенд:]
	JavaScript, Angular2+. Так же есть опыт с GWT и Flex. IDE --- ранее --- Eclipse, сейчас --- IntelliJ IDEA, WebStorm.

\item[Мобильные приложения:]
	Обычно использую что-то для быстрой разработки: Flex/AIR, Processing, React Native. Есть опыт с Android studio. 
	Мобильные приложения обычно пишу в качестве быстрого и дешевого интерфейса с гаджетами.

\item[DevOps:]
	Для веба собираю в докер, для встраиваемых решений в дебиановские пакеты и репозитории. Для сборки использую Maven и CMake.
	Есть опыт с Jenkins, немного с Ansible (факты, плейбуки). Там, где нормальные люди используют Python, предпочитаю BASH.
	
\item[{Хранилища}:]
	SQLite, MySQL, MongoDB, ElasticSearch.

\item[Тесты:]
	JUnit, Mockito, Catch2
\end{description}

\section*{Подписывайтесь, ставьте лайки}
\begin{itemize}
\item\url{https://www.linkedin.com/in/shadowlamer/}
\item\url{https://shadowlamer.github.io/}
\item\url{https://github.com/shadowlamer/}
\item\url{https://easyeda.com/shadowlamer/}
\end{itemize}
	
\end{document}