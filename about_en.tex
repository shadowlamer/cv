\documentclass [a4paper,10pt]{article}
\usepackage[T1]{fontenc}
\usepackage[utf8]{inputenc}
\usepackage[russian]{babel}
\usepackage[hidelinks]{hyperref}
\usepackage{marvosym}


\renewcommand{\labelitemi}{}

\begin{document}
\begin{center}
{\scshape\LARGE Vadim \par}
\Letter\hspace{6pt}\href{mailto:sl@anhot.ru}{sl@anhot.ru},
\Mobilefone\hspace{6pt}\href{tel:+79638704302}{+7 963 870 43 02}

developer
\end{center}
\section*{Notable experience}
\paragraph{Successful\protect\footnote{survived and continue to exist} startups}
\begin{description}
\item[<<Tretiy kran>>] \url{https://3voda.ru/} (2017 - 2020) 

I was fully responsible for all electronics and software..

\item[<<INSYTE Electronics>>] \url{https://insyte.ru/} (2003 -- 2008)

It started from my coursework on data transfer over power line. In 2007-
2008 assisted in the work directly. Insisted on shift to <<honest>> ModBus.
Elicited idempotented implementation of the function. Developed several
devices of <<LanDrive2>> line. Gave a kick-start to OEM manufacturing,
completed documentation, test beds. Provided technical support.
\end{description}

\paragraph{Not so sucessful startups}
\begin{description}
\item[<<Magicscope>>] \url{https://magicscope.com} (since 2009)

Read here for more information --- \url{https://perm.aif.ru/society/people/youtube_otdyhaet_permyaki_sozdali_novuyu_mediaplatformu}
	
Made by me --- software design, everything concerning media server and
video translation, partly back-end, all front-end and mobile application...
\end{description}

\paragraph{Team work}
\begin{description}
\item[<<PARMA Technologies Group>>] \url{https://parma.ru/} (2016 --2017)
	
Web, enterprise, full stack. Did what I was told. Took part in a dozen
of projects. <<To the Far East>> (На Дальний восток) is one of officially
mentioned. Others of the same level.

\item[<<Regional Center of Automation>>] (2008 -- 2009)

Tried to create own development shop with colleagues. We were engaged in video surveillance, vibration monitoring, industrial automation, some web.
\end{description}

\paragraph{Teaching}
\begin{description}
\item[PSIAC] \url{http://psiac.ru/} (2009 -- 2013)

Senior teacher of IIT department. Gave lectures and lab sessions in multimedia
technologies and physics.

\item[PSTU] \url{https://pstu.ru/} (2004 - 2007)

Teaching assistant of ITAS department. Gave lab sessions in various computer
science disciplines, took part in faculty RD and was at work with Russian-
Austrian feildbus\footnote{then it was not mainstream, but
	now it's called <<internet of things>>} technology center.
\end{description}

\section*{I apply this}
\begin{description}
\item[Ectronics:]
    First used PCAD, in mid-2000 switched to Eagle, tempted by platform-independence and rich component library. Now mostly use EasyEDA. Acquainted with KiCAD. It goes without saying that I use oscillograph, solderer, rasp. Usually make solutions on STM32, previously AVR, analog devices is also not a problem.

\item[Embedded software:]
	C, C++, BASH. As IDE I used Code::Blocks, Eclipse, Keil uVision at different times. Had occasion to use IAR. Now hooked on Clion.

\item[System software:]
	Linux (mostly Debian), and everything that follows(C, C++, BASH). Can build distribution for custom hardware from mud and straw. Had some positive experience in writing device drivers, bootloader.

\item[Application software:]
	I still haven’t made up my mind what technology is more suitable for quick desktop application developing. Used Delphi/Kylix/Lazarus, Electron. Now trying Qt.

\item[Web/backend:]
	Java, Spring, Spring boot. IDE --- earlier --- Eclipse, now --- IntelliJ IDEA.

\item[Web/frontend:]
	JavaScript, Angular2+. I also have experience working with GWT and Flex. IDE --- earlier --- Eclipse, now --- IntelliJ IDEA, WebStorm.

\item[Mobile apps:]
    I usually use something for rapid development: Flex / AIR, Processing. I have experience with Android studio and React Native.
    I usually write mobile applications as a fast and cheap interface with stom devices.

\item[DevOps:]
    I build docker containers for web applictions, Debian packages and repositories for embedded solutions. I use Maven and CMake for assembly. Have experience working with Jenkins, a little with Ansible (facts, playbooks). I prefer BASH where normal people use Python.
	
\item[{Хранилища}:]
	SQLite, MySQL, MongoDB, ElasticSearch.

\item[Тесты:]
	JUnit, Mockito, Catch2
\end{description}

\section*{Subscribe and like}
\begin{itemize}
\item\url{https://www.linkedin.com/in/shadowlamer/}
\item\url{https://shadowlamer.github.io/}
\item\url{https://github.com/shadowlamer/}
\item\url{https://easyeda.com/shadowlamer/}
\end{itemize}
	
\end{document}